\documentclass[a4paper]{article}

\usepackage{INTERSPEECH2021}

\title{Sign Language Alphabet Translator Using Transfer Learning with Object Detection}
\name{Yi-Chuan Leuker, Agostino Calamia, Torsten Wolter}
\email{yi-chuan.leuker@fom-net.de, agostino.calamia@fom-net.de, torsten.wolter@fom-net.de}

\begin{document}

\maketitle
% 
\begin{abstract}
  Every country has its own sign language. There is no universal sign language in the world to help dead people communicate with others from other countries. This project focuses on developing a Sign Language Alphabet (SLA) translator that can play an important role by not only its interpretation but also helping deaf people to communicate with each other without learning a new sign language. In this project, the sign language used for training detective models is American sign language, and the detective models were imported from Pre-trained models for Transfer learning in the Keras library. The best six models, VGG16, ResNet50V2, MobileNet, MobileNetV2, DenseNet201, and Xception were selected by comparing the performance of the partial learning for further training of the full training dataset. After the training of the full training dataset, the six models were compared for accuracy and inference time and the result was MobileNet with an inference time as fast as 0.03 seconds and an accuracy of 99.68\%. The test dataset was further detected using this MobileNet model and the detected results were mapped to Turkish sign language images for translation.

  Index terms should be included as shown below.
\end{abstract}
\noindent\textbf{Index Terms}: sign language alphabet recognition, transfer learnings, object detection

\section{Introduction}
World Health Organization (WHO) projected that nearly 2.5 billion people in the world have hearing loss by 2050 and emphasized that sign language-related applications are essential tools for deaf people \cite{WHO2050}.

Instead of oral language using speaking and listening, Sign language is a form of visual communication through gestures, body movements and facial expressions. Sign language can be used for different purposes in different situations, but it is primarily designed for communication with the deaf. As it is more visually accessible to the deaf, sign language is a natural method of communication for the deaf and can express a wide range of meanings in the same way as spoken language; for example, American Sign Language is used by deaf people in the United States and partial provinces of Canada, British Sign Language is used by deaf people in the UK, French Sign Language in France, Japanese Sign Language in Japan and Chinese Sign Language in China.

Although there is no unified international sign language yet, the major sign language systems such as the USA, British, France and China have developed fingerspelling, also called "Sign Language Alphabet (SLA)." SLA represents the 26 common alphabets, A to Z, and local special letters using only hand gestures, mostly in difficult words such as personal names, place names and technical terms, etc. Additionally, it can also be used to facilitate easier communication when encountering issues with sign language dialects. 

In the past decade, deep learning has shown excellent performance in image recognition, enabling breakthroughs in sign language recognition technology. Many studies have used neural networks to convert static SLA signs into text or speech, using images of hand gestures as input, and the main approach includes data acquisition technique, static signs, classification with over 90\% accuracy of recognition \cite{wadhawan2021sign}. Moreover, Transfer Learning is one of the major milestones in Deep Learning for Object Detection. With Transfer Learning, pre-trained models have already been trained on a different task, therefore it is a short-cut that re-uses the pre-trained model and shorter the training process. This project aims to utilize pre-trained models from Transfer Learning with Object Detection in Keras and to apply static hand gesture images into different pre-trained model architecture to present a development of an SLA translator, which translates American SLA to Turkish SLA, thereby helping deaf people to communicate with each other without learning a new sign language. The remainder of this paper will include related work in the Sign language recognition research area, the problem statement found in the research, the objective of the project, and the methodology applied in the project. Afterwards, six pre-trained models, VGG16, ResNet50V2, MobileNet, MobileNetV2, DenseNet201, and Xception from Transfer Learning with Object Detection in Keras library are selected and trained as predictive models. The results of relevant inference time and accuracy are evaluated as performance to determine the best model for the use case. In end, the best model predicts ASL images using the test dataset, and the predicted results have a further mapping with Turkish SLA images.
  
\section{Related Work}
SLA translator is highly influenced by hand gesture recognition research using various devices for decades. There are two main approaches of sign language recognition: sensor-based or vision-based \cite{Cheok2019ARO}. The major difference between sensor-based and vision-based approaches is on the data acquisition phase. Sensor-based approaches utilize sensor instruments such as sensory glove to capture sign language, but such equipment was too complex and expensive to be widely actual used. On the other hand, vision-based approaches don't require complex facilities to acquire data, acquiring images or videos of the sign language through camera. For example, D., Cao et al.\cite{7301347} in 2015 developed sign language recognition by adapting Microsoft Kinect technology and used Random forest to successfully recognized static 24 American SLA with above 90\% accuracy. Furthermore, A., Joshi et al.\cite{8088212} presented a real-time automated American SLA translator that translates American SLA to English text by applying edge detection and cross-correlation methodologies, resulting in 94.23\% accuracy for alphabets. In recent years, Convolutional neural network (CNN) has become a common method applied in image recognition and classification. M., Taskiran et al.\cite{8441304} designed a real-time sign language system with an implementation of feature extraction and classifier based on a CNN structure, resulted in 98.05\% accuracy.

\section {Research question}
\subsection{Problem Statement}
The previous works showed great achievements in translating American SLA to texts by capturing images through cameras and using CNN for feature extraction and classification. However, an SLA translator from American SLA to other SLA is still not available today.

\subsection{Objective}
The goal for this research question aims to develop an SLA translator that use optimal transfer learning for object detection to recognise images of American SLAs and translate them into Turkish SLAs with images. For example, the American alphabet P in sign language is recognised and translated into the Turkish alphabet P in sign language. With sign language recognition, deaf people from different countries can communicate with each other without having to learn a new sign language.

\section{Methodology}
Authors should observe the following rules for page layout. A highly recommended way to meet these requirements is to use a given template (Microsoft Word\textregistered\ or \LaTeX) and check details against the corresponding example PDF file. Given templates, Microsoft Word\textregistered\ or \LaTeX, can be adapted/imported easily in other software such as LibreOffice, Apple Pages, Lua\LaTeX, and Xe\LaTeX, but please be careful to match the layout of the provided PDF example, 

\subsection{Basic layout features}

\begin{itemize}
\item Proceedings will be printed in DIN A4 format. Authors must submit their papers in DIN A4 format.
\item Two columns are used except for the title section and for large figures that may need a full page width.
\item Left and right margin are 20 mm each. 
\item Column width is 80 mm. 
\item Spacing between columns is 10 mm.
\item Top margin is 25 mm (except for the first page which is 30 mm to the title top).
\item Bottom margin is 35 mm.
\item Text height (without headers and footers) is maximum 235 mm.
\item Headers and footers must be left empty.
\item Check indentations and spacings by comparing to this example file (in PDF).
\end{itemize}

\subsubsection{Headings}

Section headings are centered in boldface with the first word capitalized and the rest of the heading in lower case. Sub- headings appear like major headings, except they start at the left margin in the column. Sub-sub-headings appear like sub-headings, except they are in italics and not boldface. See the examples in this file. No more than 3 levels of headings should be used.

\subsection{Text font}

Times or Times Roman font is used for the main text. Font size in the main text must be 9 points, and in the References section 8 points. Other font types may be used if needed for special purposes. It is VERY IMPORTANT that while making the final PDF file, you embed all used fonts! To embed the fonts, you may use the following instructions:
\begin{enumerate}
\item For Windows users, the bullzip printer can convert any PDF to have embedded and subsetted fonts.
\item For Linux/Mac users, you may use \\
   pdftops file.pdf\\
   pstopdf -dPDFSETTINGS=/prepress file.pdf
\end{enumerate}

\LaTeX users: users should use Adobe Type 1 fonts such as Times or Times Roman. These are used automatically by the INTERSPEECH2021.sty style file. Authors must not use Type 3 (bitmap) fonts.

\subsection{Figures}

All figures must be centered on the column (or page, if the figure spans both columns). Figure captions should follow each figure and have the format given in Figure~\ref{fig:speech_production}.

Figures should be preferably line drawings. If they contain gray levels or colors, they should be checked to print well on a high-quality non-color laser printer.

Graphics (i.\,e., illustrations, figures) must not use stipple fill patterns because they will not reproduce properly in Adobe PDF. Please use only SOLID FILL COLORS.

Figures which span 2 columns (i.\,e., occupy full page width) must be placed at the top or bottom of the page.

\subsection{Tables}

An example of a table is shown in Table~\ref{tab:example}. The caption text must be above the table.

\begin{table}[th]
  \caption{This is an example of a table}
  \label{tab:example}
  \centering
  \begin{tabular}{ r@{}l  r }
    \toprule
    \multicolumn{2}{c}{\textbf{Ratio}} & 
                                         \multicolumn{1}{c}{\textbf{Decibels}} \\
    \midrule
    $1$                       & $/10$ & $-20$~~~             \\
    $1$                       & $/1$  & $0$~~~               \\
    $2$                       & $/1$  & $\approx 6$~~~       \\
    $3.16$                    & $/1$  & $10$~~~              \\
    $10$                      & $/1$  & $20$~~~              \\
    $100$                     & $/1$  & $40$~~~              \\
    $1000$                    & $/1$  & $60$~~~              \\
    \bottomrule
  \end{tabular}
  
\end{table}

\subsection{Equations}

Equations should be placed on separate lines and numbered. Examples of equations are given below. Particularly,
% 
\begin{equation}
  x(t) = s(f_\omega(t))
  \label{eq1}
\end{equation}
% 
where \(f_\omega(t)\) is a special warping function
% 
\begin{equation}
  f_\omega(t) = \frac{1}{2 \pi j} \oint_C 
  \frac{\nu^{-1k} \mathrm{d} \nu}
  {(1-\beta\nu^{-1})(\nu^{-1}-\beta)}
  \label{eq2}
\end{equation}
% 
A residue theorem states that
% 
\begin{equation}
  \oint_C F(z)\,\mathrm{d}z = 2 \pi j \sum_k \mathrm{Res}[F(z),p_k]
  \label{eq3}
\end{equation}
% 
Applying (\ref{eq3}) to (\ref{eq1}), it is straightforward to see that
% 
\begin{equation}
  1 + 1 = \pi
  \label{eq4}
\end{equation}

Finally we have proven the secret theorem of all speech sciences. No more math is needed to show how useful the result is!

\begin{figure}[t]
  \centering
  \includegraphics[width=\linewidth]{figure.pdf}
  \caption{Schematic diagram of speech production.}
  \label{fig:speech_production}
\end{figure}

\subsection{Information for Word users only}

For ease of formatting, please use the styles listed in Table 2. The styles are defined in this template file and are shown in the order in which they would be used when writing a paper. When the heading styles in Table 2 are used, section numbers are no longer required to be typed in because they will be automatically numbered by Word. Similarly, reference items will be automatically numbered by Word when the ``Reference'' style is used.

\begin{table}[t]
  \caption{Main predefined styles in Word}
  \label{tab:word_styles}
  \centering
  \begin{tabular}{ll}
    \toprule
    \textbf{Style Name}      & \textbf{Entities in a Paper}                \\
    \midrule
    Title                    & Title                                       \\
    Author                   & Author name                                 \\
    Affiliation              & Author affiliation                          \\
    Email                    & Email address                               \\
    AbstractHeading          & Abstract section heading                    \\
    Body Text                & First paragraph in abstract                 \\
    Body Text Next           & Following paragraphs in abstract            \\
    Index                    & Index terms                                 \\
    1. Heading 1             & 1\textsuperscript{st} level section heading \\
    1.1 Heading 2            & 2\textsuperscript{nd} level section heading \\
    1.1.1 Heading 3          & 3\textsuperscript{rd} level section heading \\
    Body Text                & First paragraph in section                  \\
    Body Text Next           & Following paragraphs in section             \\
    Figure Caption           & Figure caption                              \\
    Table Caption            & Table caption                               \\
    Equation                 & Equations                                   \\
    \textbullet\ List Bullet & Bulleted lists                              \\\relax
    [1] Reference            & References                                  \\
    \bottomrule
  \end{tabular}
\end{table}

If your Word document contains equations, you must not save your Word document from ``.docx'' to ``.doc'' because when doing so, Word will convert all equations to images of unacceptably low resolution.

\subsection{Hyperlinks}

For technical reasons, the proceedings editor will strip all active links from the papers during processing. Hyperlinks can be included in your paper, if written in full, e.\,g.\ ``http://www.foo.com/index.html''. The link text must be all black. 
Please make sure that they present no problems in printing to paper.

\subsection{Multimedia files}

The INTERSPEECH organizing committee offers the possibility to submit multimedia files. These files are meant for audio-visual illustrations that cannot be conveyed in text, tables and graphs. Just like you would when including graphics, make sure that you have sufficient author rights to the multimedia materials that you submit for publication. The proceedings media will NOT contain readers or players, so be sure to use widely accepted file formats, such as MPEG, Windows WAVE PCM (.wav) or Windows Media Video (.wmv) using standard codecs.

Your multimedia files must be submitted in a single ZIP file for each separate paper. Within the ZIP file you can use folders and filenames to help organize the multimedia files. In the ZIP file you should include a TEXT or HTML index file which describes the purpose and significance of each multimedia file. From within the manuscript, refer to a multimedia illustration by its filename. Use short file names without blanks for clarity.

The ZIP file you submit will be included as-is in the proceedings media and will be linked to your paper in the navigation interface of the proceedings. Causal Productions (the publisher) and the conference committee will not check or change the contents of your ZIP file.

Users of the proceedings who wish to access your multimedia files will click the link to the ZIP file which will then be opened by the operating system of their computer. Access to the contents of the ZIP file will be governed entirely by the operating system of the user's computer.

\subsection{Page numbering}

Final page numbers will be added later to the document electronically. \emph{Do not make any footers or headers!}


\subsection{References}
The reference format \cite{Wiederhofer2017} is the standard IEEE one. References should be numbered in order of appearance, for example \cite{Davis80-COP}, \cite{Rabiner89-ATO}, \cite[pp.\ 417--422]{Hastie09-TEO}, and \cite{YourName21-XXX}.

\subsection{Abstract}

The total length of the abstract is limited to 200 words. The abstract included in your paper and the one you enter during web-based submission must be identical. Avoid non-ASCII characters or symbols as they may not display correctly in the abstract book.

\subsection{Author affiliation}

Please list country names as part of the affiliation for each country.

\subsection{Number of authors in the author list}

The maximum number of authors in the author list is twenty. If the number of contributing authors is more than twenty, they should be listed in a footnote or in acknowledgement section, as appropriate.

\subsection{Submitted files}

Authors are requested to submit PDF files of their manuscripts. You can use commercially available tools or for instance http://www.pdfforge.org/products/pdfcreator. The PDF file should comply with the following requirements: (a) there must be no PASSWORD protection on the PDF file at all; (b) all fonts must be embedded; and (c) the file must be text searchable (do CTRL-F and try to find a common word such as ``the''). The proceedings editors (Causal Productions) will contact authors of non-complying files to obtain a replacement. In order not to endanger the preparation of the proceedings, papers for which a replacement is not provided in a timely manner will be withdrawn.

\section{Discussion}

This is the discussion. This is the discussion. This is the discussion. Is there any discussion?

Lorem ipsum dolor sit amet, consectetur adipiscing elit. Cras consequat mollis odio, nec venenatis enim auctor sed. Integer tincidunt fringilla lectus eget condimentum. In eget sapien id eros dapibus interdum vel ac quam. Aenean vitae rutrum erat. Aenean et risus pharetra, lacinia augue ut, fermentum ante. Integer dui arcu, interdum at ornare a, faucibus quis est. Mauris quis quam felis. Etiam pulvinar massa et turpis lacinia, eu posuere mi iaculis. Fusce at velit quis leo dignissim porttitor.

Fusce ut nunc eu sapien venenatis finibus a vel ligula. Pellentesque habitant morbi tristique senectus et netus et malesuada fames ac turpis egestas. Ut quam eros, volutpat at gravida consectetur, rutrum ut leo. Aenean cursus euismod feugiat. Cras hendrerit, ligula eu feugiat malesuada, neque turpis auctor lacus, sit amet accumsan neque orci a quam. Mauris suscipit ultrices mattis. Nulla at interdum metus, id pharetra diam. Curabitur at vestibulum sem, sed elementum massa. Donec iaculis et arcu ut rutrum. Fusce gravida, mauris porta volutpat eleifend, enim mauris eleifend orci, eu ultrices leo purus vitae metus. In pretium dolor ut magna dictum, at imperdiet lectus porta.

Quisque mollis lectus id risus pretium mattis. Morbi scelerisque posuere est, id efficitur urna luctus non. Praesent quam lacus, facilisis id ante eu, vehicula maximus ex. Nullam mollis in arcu vitae efficitur. Aliquam molestie eleifend ante, in pretium velit ultrices ac. Etiam laoreet nec sem non pulvinar. Integer ligula felis, interdum non lacus id, malesuada imperdiet turpis.

Aenean sit amet volutpat nisi. Aliquam eu erat quis tortor ultrices laoreet. Vivamus fermentum semper metus, non faucibus libero euismod vitae. Sed efficitur porta congue. Aenean in faucibus nisi. Donec suscipit augue vitae orci consequat, sit amet aliquet felis varius. Duis efficitur lacinia dolor sit amet lobortis. Curabitur erat sapien, molestie nec nisi eu, dignissim accumsan ipsum. Fusce id nibh nec risus dictum posuere in ac magna. Donec malesuada massa sed erat lacinia cursus. Suspendisse ornare augue nec volutpat consequat.

Vestibulum et vulputate nisi, a malesuada mi. Nam pellentesque arcu sapien, at placerat odio imperdiet ut. Curabitur nec venenatis tellus, vel aliquet nisi. Curabitur vel ligula sit amet metus auctor pretium. Nullam nulla mi, blandit a mattis id, vulputate sit amet enim. Proin mollis fringilla dictum. Proin lacinia orci purus.

Curabitur porttitor bibendum dolor, nec consectetur sapien pulvinar id. Donec eleifend, est vel dignissim pretium, tortor augue euismod nunc, id fermentum erat felis ac neque. Morbi id lectus ultricies, rutrum justo eu, sollicitudin risus. Suspendisse lobortis efficitur nisi sit amet pellentesque. Ut eget augue at mi aliquet mattis. Proin et feugiat erat, sit amet sodales eros. Integer sed elit quis est mattis ullamcorper. Pellentesque lectus nisi, vulputate a imperdiet tincidunt, auctor nec orci. Pellentesque sagittis nisl orci, vitae placerat massa lacinia nec. Sed egestas magna sed augue sollicitudin luctus. Praesent interdum bibendum tortor, eu porta purus. Aliquam convallis velit id mi fermentum, sed ornare eros cursus.

Quisque congue leo a fringilla pharetra. Phasellus sed tempor est, sed auctor purus. Morbi vel lacus ullamcorper, auctor mauris id, pulvinar lorem. Suspendisse potenti. Nam porta, purus non eleifend bibendum, erat metus pellentesque elit, non luctus nibh nunc ornare nisl. Sed rutrum lacinia nisi ac suscipit. Curabitur non blandit augue. Integer viverra, ipsum vel molestie euismod, sem quam tempus massa, eget efficitur ante turpis non metus. Quisque efficitur posuere velit in iaculis. Cras imperdiet varius urna vitae vestibulum. Donec accumsan eget nisi sed pellentesque. Vestibulum id quam ut urna volutpat ullamcorper gravida sit amet libero. Aliquam bibendum, ligula vitae porta malesuada, arcu diam congue erat, a pharetra diam sem vulputate tortor. Etiam luctus iaculis leo cursus tristique.

Mauris mattis sem dolor, sit amet ullamcorper arcu tincidunt ac. Vestibulum at blandit tortor. Quisque bibendum congue leo, vitae eleifend massa. Vestibulum vitae odio elit. Lorem ipsum dolor sit amet, consectetur adipiscing elit. Ut sagittis quam vel felis ornare, in gravida felis tempor. Donec molestie dui quis leo venenatis blandit. Nunc sit amet finibus metus. Cras ut lectus ex.

Suspendisse commodo libero vel leo tincidunt, a tempus mauris porta. Integer varius eros ac sapien lacinia vehicula. Donec porttitor, lacus faucibus rhoncus venenatis, neque quam imperdiet nunc, id consectetur metus purus quis sapien. Phasellus interdum nulla vel euismod posuere. Vestibulum finibus magna vel finibus mollis. Curabitur mollis turpis tortor, hendrerit vulputate justo egestas quis. Nam dignissim luctus leo non elementum. Phasellus a metus at leo malesuada bibendum. Mauris quis eleifend magna, nec vehicula ex. Donec venenatis urna fermentum commodo vehicula. Ut mattis scelerisque aliquam. Vivamus pulvinar erat metus, id tempus mi vulputate quis.

Fusce lobortis a urna eget blandit. Vivamus in eleifend neque, at sollicitudin lectus. Quisque faucibus egestas lorem, in commodo diam maximus eu. Morbi finibus ante ac felis porttitor euismod. Donec lobortis aliquam ipsum sit amet luctus. Cum sociis natoque penatibus et magnis dis parturient montes, nascetur ridiculus mus. Etiam rutrum neque sapien, eget luctus turpis iaculis pulvinar. Duis quis pulvinar nunc, nec bibendum ligula. Phasellus suscipit sagittis lacus molestie laoreet. Pellentesque lacus diam, tincidunt a aliquam vitae, aliquet non justo.

Etiam lectus lacus, commodo eget consectetur eget, auctor vitae leo. Praesent vitae erat in diam blandit semper vitae et eros. Maecenas auctor pharetra nibh eget egestas. Donec accumsan ut risus eget rhoncus. Nam placerat, erat sit amet gravida mollis, purus arcu accumsan diam, tempus pharetra risus mi ac sapien. Ut et tortor porta, pulvinar elit vitae, tempor mi. Nam interdum, nisl non pharetra molestie, turpis neque commodo ligula, sit amet pretium nisl nibh quis ante. Quisque et ex eget velit lobortis suscipit. Integer aliquam finibus molestie. Sed pellentesque neque eu turpis aliquet, mattis ornare enim finibus. In hac habitasse platea dictumst.

Integer congue quis justo a posuere. Quisque porta, ante et dignissim suscipit, arcu mauris ultrices libero, nec sollicitudin purus lacus a enim. Aliquam feugiat eget lacus molestie sodales. Duis blandit placerat nunc, et venenatis turpis dictum vel. Nulla facilisi. Nam ullamcorper, tellus eu posuere mattis, arcu lacus dictum nulla, vel mattis nisi sem posuere tellus. Etiam quis eros condimentum lectus lobortis eleifend. In ex lacus, sodales scelerisque egestas ac, aliquam nec purus. Nunc sit amet magna non libero ullamcorper dictum. Phasellus porta faucibus tempus. Praesent blandit tortor sed tellus ornare consectetur. Sed sed nisi id neque porta varius eu eu velit. Curabitur varius convallis justo id facilisis. Mauris auctor velit nec aliquam cursus.

Integer suscipit scelerisque leo sed faucibus. Ut commodo nulla luctus diam posuere egestas. Integer ut augue ac velit ullamcorper tempus. Pellentesque in mi rhoncus, sodales sem quis, commodo sem. Aenean dapibus euismod diam id rhoncus. Nullam vehicula placerat eros consectetur luctus. Aliquam auctor ipsum vitae egestas imperdiet. Ut nulla lacus, imperdiet quis urna vel, ornare imperdiet tortor. Mauris nec diam ac nunc laoreet volutpat at id turpis. Nulla eu neque a risus feugiat iaculis ac vel risus. Ut tempus elementum lorem eget porta. Nullam et ullamcorper urna. Cum sociis natoque penatibus et magnis dis parturient montes, nascetur ridiculus mus. Phasellus eget dui vitae nulla hendrerit ultrices quis rutrum leo.

Proin consectetur lacus sit amet eleifend varius. Etiam eu blandit risus. Curabitur pellentesque urna sed dolor congue mattis. Vestibulum ut velit posuere, feugiat leo a, dignissim massa. Proin eu nulla risus. Fusce luctus bibendum est, sit amet venenatis nisi finibus non. Donec ultricies ornare nunc at lobortis. Nam a auctor metus. Vestibulum pretium condimentum turpis ac mattis. Curabitur semper sagittis rhoncus. Duis molestie facilisis mattis. Sed pharetra lorem id tortor efficitur, sed maximus leo posuere. Quisque suscipit molestie convallis. Duis imperdiet placerat congue. Morbi placerat, velit ut tempor porta, ex nisi imperdiet purus, non feugiat ex velit nec nulla.

Phasellus mattis at erat eget lobortis. Vivamus sodales odio non erat luctus faucibus. Curabitur aliquam luctus nulla quis consectetur. Fusce vulputate finibus vulputate. Cras at condimentum massa. Duis vestibulum ipsum ac tortor lobortis fringilla. Cras non neque at nunc pellentesque mollis. Sed quis erat mauris. Ut dapibus sem lectus, quis imperdiet diam bibendum et. Maecenas quis venenatis ante. Nunc blandit a risus sed scelerisque. Praesent cursus est sit amet nisi tempus, quis placerat libero rutrum. Phasellus lacinia nisi quis consequat mollis. Phasellus sagittis aliquam lacus.

Sed dolor quam, posuere nec nunc eget, feugiat lobortis ligula. Fusce lacinia fermentum dolor, luctus dapibus ex venenatis feugiat. In hac habitasse platea dictumst. Nullam vitae ligula dignissim, interdum turpis quis, tincidunt metus. Nullam in nisl vitae mauris egestas porta. Nam fringilla aliquet sapien, non dapibus nunc sollicitudin id. Nunc hendrerit felis et vehicula consequat. Praesent varius libero id volutpat iaculis. Aliquam vel dui imperdiet, pharetra augue sed, iaculis nulla. Quisque mollis orci nec odio eleifend, eget laoreet nunc feugiat. Cras varius tortor a fringilla gravida.

Sed posuere erat eu dolor consequat euismod. Donec imperdiet, tellus nec convallis commodo, lorem sem lobortis purus, a lacinia massa ipsum vitae nisl. Vivamus auctor tellus in urna iaculis luctus. In dui nibh, posuere a erat a, lobortis finibus nulla. Sed vel suscipit nisi. Nunc eget nibh risus. Sed posuere tempus eleifend. Nullam ac lacinia ligula, ut blandit erat. In at est sed turpis consequat rutrum. Nunc eget lectus venenatis, convallis ligula non, mollis orci. Aliquam sit amet ligula turpis. Sed finibus laoreet elit nec molestie.

Mauris in nisi et neque euismod aliquam ut eget felis. Sed eget dictum tellus, finibus rutrum nibh. Fusce placerat augue a faucibus semper. Nam sed nisl ligula. Vivamus ante augue, faucibus at risus ut, hendrerit viverra risus. Etiam justo eros, dignissim a nunc sed, porta mollis erat. Fusce in sem accumsan, laoreet nibh porta, porttitor orci. Donec venenatis vehicula ante eget dictum.

Aliquam fermentum a metus pellentesque cursus. Vivamus eleifend ultricies tellus, vel scelerisque diam. Sed hendrerit est at elit suscipit placerat. Praesent nec pretium erat. Quisque blandit nunc in felis bibendum consequat. 
%Aliquam quis orci consectetur nulla luctus ullamcorper. Suspendisse finibus luctus erat a dapibus.

\section{Conclusions}

Authors must proofread their PDF file prior to submission to ensure it is correct. Authors should not rely on proofreading the Word file. Please proofread the PDF file before it is submitted.

\section{Acknowledgements}

The ISCA Board would like to thank the organizing committees of the past INTERSPEECH conferences for their help and for kindly providing the template files. \\
Note to authors: Authors should not use logos in the acknowledgement section; rather authors should acknowledge corporations by naming them only.


\bibliographystyle{IEEEtran}

\bibliography{mybib}

% \begin{thebibliography}{9}
% \bibitem[1]{Davis80-COP}
%   S.\ B.\ Davis and P.\ Mermelstein,
%   ``Comparison of parametric representation for monosyllabic word recognition in continuously spoken sentences,''
%   \textit{IEEE Transactions on Acoustics, Speech and Signal Processing}, vol.~28, no.~4, pp.~357--366, 1980.
% \bibitem[2]{Rabiner89-ATO}
%   L.\ R.\ Rabiner,
%   ``A tutorial on hidden Markov models and selected applications in speech recognition,''
%   \textit{Proceedings of the IEEE}, vol.~77, no.~2, pp.~257-286, 1989.
% \bibitem[3]{Hastie09-TEO}
%   T.\ Hastie, R.\ Tibshirani, and J.\ Friedman,
%   \textit{The Elements of Statistical Learning -- Data Mining, Inference, and Prediction}.
%   New York: Springer, 2009.
% \bibitem[4]{YourName17-XXX}
%   F.\ Lastname1, F.\ Lastname2, and F.\ Lastname3,
%   ``Title of your INTERSPEECH 2021 publication,''
%   in \textit{Interspeech 2021 -- 20\textsuperscript{th} Annual Conference of the International Speech Communication Association, September 15-19, Graz, Austria, Proceedings, Proceedings}, 2020, pp.~100--104.
% \end{thebibliography}

\end{document}
