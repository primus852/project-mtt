\documentclass[a4paper]{article}

\usepackage{INTERSPEECH2021}

\title{Sign Language Alphabet Translator Using Transfer Learning with Object Detection}
\name{Yi-Chuan Leuker, Agostino Calamia, Torsten Wolter}
\email{yi-chuan.leuker@fom-net.de, agostino.calamia@fom-net.de, torsten.wolter@fom-net.de}

\begin{document}

\maketitle
%
\begin{abstract}
  Every country has its own sign language. There is no universal sign language in the world to help deaf people communicate with others from other countries. This project focuses on developing a Sign Language Alphabet (SLA) translator that can play an important role by not only its interpretation but also helping deaf people to communicate with each other without learning a new sign language. In this project, the sign language used for training detective models is American sign language, and the detective models were imported from pre-trained models for transfer learning in the Keras library. The best six models, VGG16, ResNet50V2, MobileNet, MobileNetV2, DenseNet201, and Xception were selected by comparing the performance of the partial learning to further use these models for training of the entire training dataset. After the training of the full training dataset, the six models were compared for accuracy and inference time and the result was MobileNet with an inference time as fast as 0.03 seconds and an accuracy of 99.68\%. The test dataset was further detected using this MobileNet model and the detected results were mapped to Turkish sign language images for translation.

  Index terms should be included as shown below.
\end{abstract}
\noindent\textbf{Index Terms}: sign language alphabet recognition, transfer learnings, object detection

\section{Introduction}
The World Health Organization (WHO) projected that nearly 2.5 billion people in the world have hearing loss by 2050 and emphasized that sign language-related applications are essential tools for deaf people \cite{WHO2050}.

Instead of oral language using speaking and listening, Sign language is a form of visual communication through gestures, body movements and facial expressions. Sign language can be used for different purposes in different situations, but it is primarily designed for communication with the deaf. As it is more visually accessible to the deaf, sign language is a natural method of communication for the deaf people and can express a wide range of meanings in the same way as spoken language; for example, American Sign Language is used by deaf people in the United States and partial provinces of Canada, British Sign Language is used by deaf people in the UK, French Sign Language in France, Japanese Sign Language in Japan and Chinese Sign Language in China.

Although there is no unified international sign language yet, the major sign language systems such as the USA, British, France and China have developed fingerspelling, also called "Sign Language Alphabet" (SLA). SLA represents the 26 common alphabets, A to Z, and local special letters using only hand gestures, mostly in difficult words such as personal names, place names and technical terms, etc. Additionally, it can also be used to facilitate easier communication when encountering issues with sign language dialects. 

In the past decade, deep learning has shown excellent performance in image recognition, enabling breakthroughs in sign language recognition technology. Many studies have used neural networks to convert static SLA signs into text or speech, using images of hand gestures as input, and the main approach includes data acquisition technique, static signs, classification with over 90\% accuracy of recognition \cite{wadhawan2021sign}. Moreover, Transfer Learning is one of the major milestones in deep Learning for Object Detection. With transfer learning, pre-trained models have already been trained on a different task, therefore it is a short-cut that re-uses the pre-trained model such as its trained weights for a shorter training process. This project aims to utilize pre-trained models from transfer learning with Object Detection in Keras and to apply static hand gesture images into different pre-trained model architecture to present a development of an SLA translator, which translates American SLA to Turkish SLA, thereby helping deaf people to communicate with each other without learning a new sign language. The remainder of this paper will include related work in the Sign language recognition research area, the problem statement found in the research, the objective of the project, and the methodology applied in the project. Afterwards, six pre-trained models, VGG16, ResNet50V2, MobileNet, MobileNetV2, DenseNet201, and Xception from transfer learning with Object Detection in Keras are selected and trained as predictive models. The results of relevant inference time and accuracy are evaluated as final metrics to determine the best model for the use case. In end, the best model will be used to detect letters from the ASL images using the test dataset and map those to the Turkish SLA images.
  
\section{Related Work}
A SLA translator is highly influenced by hand gesture recognition research using various devices for decades. There are two main approaches of sign language recognition: sensor-based or vision-based \cite{Cheok2019ARO}. The major difference between sensor-based and vision-based approaches is on the data acquisition phase. Sensor-based approaches utilize sensor instruments such as sensory glove to capture sign language, but such equipment was too complex and expensive to be widely actual used. On the other hand, vision-based approaches don't require complex facilities to acquire data, acquiring images or videos of the sign language through camera. For example, D., Cao et al.\cite{7301347} in 2015 developed sign language recognition by adapting Microsoft Kinect technology and used Random forest to successfully recognized static 24 American SLA with above 90\% accuracy. Furthermore, A., Joshi et al.\cite{8088212} presented a real-time automated American SLA translator that translates American SLA to English text by applying edge detection and cross-correlation methodologies, resulting in 94.23\% accuracy for alphabets. In recent years, Convolutional neural network (CNN) has become a common method applied in image recognition and classification. M., Taskiran et al.\cite{8441304} designed a real-time sign language system with an implementation of feature extraction and classifier based on a CNN structure, resulted in 98.05\% accuracy.

\section {Research question}
\subsection{Problem Statement}
The previous works showed great achievements in translating American SLA to texts by capturing images through cameras and using CNN for feature extraction and classification. However, an SLA translator from American SLA to other SLA is still not available today which would solve the international communication of people. The challenge lies in identifying a detector which is delivers a high accuracy and fast predictions at the same time.

\subsection{Objective}
The goal for this research question aims to develop an SLA translator that uses optimal transfer learning for Object Detection to recognise images of American SLAs and translate them into Turkish SLAs with images. For example, the American alphabet P in sign language is recognised and translated into the Turkish alphabet P in sign language. The main focus will be to find a model that balances accuracy such as inference time.

\section{Methodology}

Figure \ref{fig:the proposed approach} shows the proposed approach to the ASL translator. The main approach to developing a translation from American SLA to other SLA is to build the best image recognition model that recognizes each English alphabet from the sign language with high accuracy. Instead of learning from scratch, Transfer learning with pre-trained models, which were already trained on a large benchmark image dataset, can save a lot of computation cost and help the performance. Kera applications provide a wide range of pre-trained models for deep learning such as VGG, ResNet, Xception, etc \cite{keras}. All the available pre-trained models will be applied into the first run training with the partial dataset to find out the Top 5 models with the highest accuracy of prediction. The top 5 models are further optimized and trained with all the training data to determine the best ASL recognition model. In the last phase, the results of prediction are mapping with Turkish ASL images.

\begin{figure}[h]
    \centering
    \caption{The proposed approach for the sign language translator}
	\label{fig:the proposed approach}
    \includegraphics[width=\linewidth]{figures/The approach}
\end{figure}


\subsection{Dataset}
The American ASL dataset is collected from Kaggle. The dataset contains 87,000 images with 26 Englich alphabets and 3 extra signs, which are delete, space, and nothing. All the images are equally distributed to the 29 signs. In other words, each sign has 4,300 images. The whole dataset is split into training and test dataset. Training dataset is 80\% of the data, and the rest data belongs to the test dataset. 


\subsection{Image preprocessing}

\subsection{Transfer learning models}
One of the main benefits of using transfer learning is to make use of previously trained models and save computational cost (CC) for basic tasks like the removal of background. It may also be used when the training data is sparse. For the detection of American SLA we may use (to a degree) any pre-trained model for object detection on our dataset of American SLA signs even if they are limited in count and quality, as they only make the last layer of our final model.

To determine, whether a model may be suitable for our application, we use all of the models and its variants available in Keras\cite{keras}, as shown in Table~\ref{tab:keras_models}. The \textit{Avg. Top-5 Accuracy} in the table refers to the average accuracy of all variants combined.
\begin{table}[th]
    \caption{Keras Applications}
    \label{tab:keras_models}
    \centering
    \begin{tabular}{@{}lrcl@{}}
    \toprule
    Name         & \multicolumn{1}{l}{\begin{tabular}[c]{@{}l@{}}Avg. Top-5\\ Accuracy\end{tabular}} & \multicolumn{1}{l}{Total} & \begin{tabular}[c]{@{}l@{}}Variants\\ (*=Model)\end{tabular}                         \\ \midrule
    Xception     & 0.945                                                                             & 1                         & *                                                                                    \\
    VGG          & 0.901                                                                             & 2                         & *16, *19                                                                             \\
    ResNet       & 0.931                                                                             & 6                         & \begin{tabular}[c]{@{}l@{}}*50, *101, *152,\\  *50V2, *101V2,\\  *152V2\end{tabular} \\
    Inception    & 0.945                                                                             & 2                         & *V3, *ResNetV2                                                                       \\
    MobileNet    & 0.898                                                                             & 2                         & *, *V2                                                                               \\
    DenseNet     & 0.930                                                                             & 3                         & *121, *169, *201                                                                     \\
    NASNet       & 0.940                                                                             & 2                         & *Mobile, *Large                                                                      \\
    EfficientNet & -                                                                                 & 8                         & \begin{tabular}[c]{@{}l@{}}*B0, *B1, *B2,\\  *B3, *B4, *B5,\\  *B6, *B7\end{tabular} \\ \bottomrule
    \end{tabular}
\end{table}

In order to determine, which of the above stated models we will further evaluate and possibly optimize, we train each of the Keras models in an experimental setup. The experimental training is defined by:
\begin{enumerate}
    \item Training with 5\% of the dataset (~4.300 images, equally distributed to the 29 targets), with a 80/20 Train-Test-Split
    \item 10 Epochs of Training
    \item No Early Stopping
\end{enumerate}

The results will focus on two key indicators: accuracy on our dataset and training time.

The following subsections will give a brief introduction of the most relevant model families and how they differ.

\subsubsection{VGG16}
The VGG models are one of the earlier computer vision models an were introduced in 2015 by Simonyan and Zisserman which showed that using deep architectures with rather small filters can be superior to other models at that time\cite{simonyan2015deep}. VGG represents a classical convolutional neural network which takes an 224x224x3 (width x height x channels) input pictures and passes it to multiple convolutional layers, pooling layers and activation functions to a fully connected final layer for classification. It includes in total 16 layers with weights and is shown in figure \ref{fig:vgg16}.

\begin{figure}
  \centering
  \includegraphics[width=\linewidth]{figures/vgg16.png}
  \caption{VGG16 architecture}
  \label{fig:vgg16}
\end{figure}

The tremendous change compared to previous convolutional neural networks is that the used filters and layers have the same size and parameters such as only ReLu as activation function\cite{simonyan2015deep}. This newly introduced simplicity paired with a relatively deep architecture led to unique results in the ILSVRC-2012 and ILSVRC-2013 competitions compared to the previous state-of-art AlexNet. On top of that is was shown that the model had a strong ability to generalise over multiple datasets and still achieve a top 5 performance. The authors of the network structured it in a way that it is possible to move from 16 layers to 19 layers for even better results.

\subsubsection{ResNet50V2}

\subsubsection{MobileNet}

\subsubsection{MobileNetV2}

\subsubsection{DenseNet 201}

\subsubsection{Xception}

\section{Results}\label{chapter_results}
In order to find the most suitable algorithm for translating a live input stream of American SLA to any other suitable SLA (sharing the same alphabet) in a real-world application, the algorithm needs to excel in two main aspects: accuracy and inference time. A third factor that may come into question here is the training time, as computational cost may accumulate when further improving the model in the future.

\section{Discussion}

\section{Conclusions}

This scientific work proposed an applied review of multiple state-of-art deep learning models to elaborate their usability for sign language translation. The specified problem of not having intense research in the field of translating letters from one sign language to another was the main motivation. The goal was to compare multiple models with regard to accuracy such as latency to make an educated guess of which of the proposed models is the most suited one for a user facing application. To validate which model to choose, a 87.000 image big dataset with American sign language letters was used as training data. Starting with a long list of 27 pre-trained models from the Keras framework a short list of only the 6 best performing models was chosen to proceed with. The models were trained by applying transfer learning to maintain the previously learned weights. All models were compared based on their accuracy and prediction time such as the number of frames per second which they can process. It was shown that the accuracy of all models was quite similar which led to the fact that no model could have been proposed solely based on this indicator. However due to the different model architectures the research showed significant difference in the prediction time such as number per frames which were processed. MobileNetV1 resulted as preferred model from this comparison as the model accuracy such as its prediction speed were superior compared to the other architectures.
Nevertheless the current workflow is only applicable for languages which share the same letters as the predicted output is only mapped to the other language's letter. Further work has to include a more sophisticated word and letter embedding for the actual translation after a correct prediction.
Also the actual performance of an implemented infrastructure has to be tested as all experiments were executed in-vitro.


\bibliographystyle{IEEEtran}

\bibliography{mybib}

% \begin{thebibliography}{9}
% \bibitem[1]{Davis80-COP}
%   S.\ B.\ Davis and P.\ Mermelstein,
%   ``Comparison of parametric representation for monosyllabic word recognition in continuously spoken sentences,''
%   \textit{IEEE Transactions on Acoustics, Speech and Signal Processing}, vol.~28, no.~4, pp.~357--366, 1980.
% \bibitem[2]{Rabiner89-ATO}
%   L.\ R.\ Rabiner,
%   ``A tutorial on hidden Markov models and selected applications in speech recognition,''
%   \textit{Proceedings of the IEEE}, vol.~77, no.~2, pp.~257-286, 1989.
% \bibitem[3]{Hastie09-TEO}
%   T.\ Hastie, R.\ Tibshirani, and J.\ Friedman,
%   \textit{The Elements of Statistical Learning -- Data Mining, Inference, and Prediction}.
%   New York: Springer, 2009.
% \bibitem[4]{YourName17-XXX}
%   F.\ Lastname1, F.\ Lastname2, and F.\ Lastname3,
%   ``Title of your INTERSPEECH 2021 publication,''
%   in \textit{Interspeech 2021 -- 20\textsuperscript{th} Annual Conference of the International Speech Communication Association, September 15-19, Graz, Austria, Proceedings, Proceedings}, 2020, pp.~100--104.
% \end{thebibliography}

\end{document}
